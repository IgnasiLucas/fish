\documentclass[a4paper,12pt]{article}
\usepackage[T1]{fontenc}
\usepackage[utf8]{inputenc}
\author{\small \underline{Lucas-Lled\'o, Jos\'e Ignacio}$^{1,2}$, Monaghan, Michael T.$^1$ \& Mehner, Thomas$^1$}
\title{Sympatric speciation or secondary contact of Baltic ciscoes?}
\date{}
\begin{document}
\maketitle
\begin{center}
{\scriptsize
\begin{tabular}{ll}
$^1$&Leibniz Institute of Freshwater Ecology and Inland Fisheries (IGB), Berlin, Germany.\\
$^2$&Institut Cavanilles de Biodiversitat i Biologia Evolutiva, Universitat de València, Spain.\\
\end{tabular}
}
\end{center}
\begin{abstract}
Adaptive radiations of coregonid fish in postglacial lakes provide some of the most compelling examples of sympatric, ecological speciation. In several Eurasian and North American lakes, pairs of closely related \emph{Coregonus} (Teleostei: Coregonidae) species coexist, exploiting different niches, and they are likely to have diverged in sympatry. However, the complex history of postglacial lakes makes it difficult to reconstruct the evolution of these populations with only a few genetic markers. One disputed case is found in two geographically close, deep lakes in Germany, where the autumn-spawning \emph{Coregonus albula} co-occurs with either \emph{C. fontanae} or \emph{C. lucinensis}, both spring spawners. The question is whether the spring-spawning species evolved independently in both lakes from \emph{C. albula} or not. Previous studies using mitochondrial, microsatellite, and AFLP markers have provided inconclusive evidence. In the present study we use double-digest restriction-associated DNA sequencing of 24 individuals to determine the best-supported relationship among the four populations, across the genome.
\end{abstract}
\end{document}
